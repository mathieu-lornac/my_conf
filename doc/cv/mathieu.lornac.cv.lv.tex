%% start of file `template_en.tex'.
%% Copyright 2006-1008 Xavier Danaux (xdanaux@gmail.com).
%
% This work may be distributed and/or modified under the
% conditions of the LaTeX Project Public License version 1.3c,
% available at http://www.latex-project.org/lppl/.

\documentclass[10pt,a4paper]{moderncv}

% moderncv themes
\moderncvtheme[green]{classic}                 % optional argument are 'blue' (default), 'orange', 'red', 'green', 'grey' and 'roman' (for roman fonts, instead of sans serif fonts)
%\moderncvtheme[green]{casual}                 % optional argument are 'blue' (default), 'orange', 'red', 'green', 'grey' and 'roman' (for roman fonts, instead of sans serif fonts)
%\moderncvtheme[green]{classic}                % idem

% character encoding
\usepackage[utf8]{inputenc}                   % replace by the encoding you are using
\usepackage{enumerate}
% adjust the page margins
\usepackage[scale=0.88]{geometry}
%\setlength{\hintscolumnwidth}{3cm}						% if you want to change the width of the column with the dates
%\AtBeginDocument{\setlength{\maketitlenamewidth}{6cm}}  % only for the classic theme, if you want to change the width of your name placeholder (to leave more space for your address details
\AtBeginDocument{\recomputelengths}                     % required when changes are made to page layout lengths

% personal data
\firstname{Lornac}
\familyname{Mathieu}
%\title{Curriculum Vitae}               % optional, remove the line if not wanted
\address{10 avenue mar\'echal Reille}
{06600 Antibes}    % optional, remove the line if not wanted
\mobile{06 22 45 37 54}                    % optional, remove the line if not wanted
%\phone{phone (optional)}                      % optional, remove the line if not wanted
%\fax{fax (optional)}                          % optional, remove the line if not wanted
\email{mathieu.lornac@gmail.com}                      % optional, remove the line if not wanted
%\extrainfo{Permis A et B} % optional, remove the line if not wanted

%\photo[64pt]{picture}                         % '64pt' is the height the picture must be resized to and 'picture' is the name of the picture file; optional, remove the line if not wanted
%\quote{Some quote (optional)}                 % optional, remove the line if not wanted

\nopagenumbers{}                             % uncomment to suppress automatic page numbering for CVs longer than one page
\usepackage{graphicx}
%----------------------------------------------------------------------------------
%            content
%----------------------------------------------------------------------------------
\begin{document}
\maketitle

\section{Exp\'eriences}

\cventry{02/13 -}{Consultant Ausy pour Orange}{}{Profile and Syndication}{Sophia Antipolis}{Environnement: C++, PHP, Ruby, NOSQL(Cassandra), Ubuntu}
\cventry{08/12 - 01/13}{Consultant Ausy pour SpotImage}{Production d'images satellites et a\'eriennes}{Evolution et maintenance d'\'editeurs d'images d\'edi\'es \`a la production de données cartographiées}{Sophia Antipolis}{Environnement: C++, Qt, CentOS}
\cventry{03/12 - 08/12}{DoremiLabs}{Cin\'ema num\'erique}{D\'eveloppement d'une solution de r\'eception de flux satellites permettant la diffusion d'\'ev\`enements en direct en salles de cin\'ema}{Sophia Antipolis}{Environnement: C/C++, CVS, Debian}
\cventry{10/11 - 03/12}{MargOconseil}{Finance des march\'es, D\'eveloppement d'un framework de g\'en\'eration de code et d'interop\'erabilit\'e de langages}{}{Paris}{Environnement: Ruby, C++, Mercurial, GNU/Linux}
\cventry{08/10 - 09/11}{Ensuite Informatique}{D\'eveloppement de processeurs d'analyse de situation en continue}{Conception, impl\'ementation}{Paris}{Environnement: C++, Qt, Flex/Bison, CMake, Git, GNU/Linux}

\bigskip
\section{Technos}
\cvcomputer{Langages}{\em{C++} \hfill \includegraphics[trim = 0mm 11mm 0mm 36mm, clip, scale=0.35]{star_rate.png}}{}{\em{C} \hfill \includegraphics[trim = 0mm 11mm 0mm 36mm, clip, scale=0.35]{star_rate.png}}
\cvcomputer{}{\em{Ruby} \hfill \includegraphics[trim = 0mm 23mm 0mm 25mm, clip, scale=0.35]{star_rate.png}}{}{\em{Php} \hfill \includegraphics[trim = 0mm 36mm 0mm 13mm, clip, scale=0.35]{star_rate.png}}
\cvcomputer{}{\em{Cmake} \hfill \includegraphics[trim = 0mm 37mm 0mm 13mm, clip, scale=0.35]{star_rate.png}}{}{\em{Java} \hfill \includegraphics[trim = 0mm 50mm 0mm 0mm, clip, scale=0.35]{star_rate.png}}
\cvcomputer{Connaissances}{\LaTeX, bash, python, RoR}{}{}

\bigskip

\section{Formation}
\cventry{2012}{Epitech}{Ecole d'expertise en informatique et nouvelles technologies}{5\`eme ann\'ee}{Paris}{}
\cventry{\'Et\'e 2008}{Stage linguistique. Universit\'e Alliant}{San Diego USA}{}{}{}
\cventry{2004--2006}{BTS Informatique de gestion}{Lyc\'ee Suzanne Valadon}{Limoges}{}{}
\cventry{2003--2004}{MIAS (Math\'ematiques et Informatique Appliqu\'es aux sciences)}
{1\`ere ann\'ee}{Facult\'e des sciences de Limoges}{}{}
\cventry{2003}{Baccalaur\'eat Scientifique sp\'ecialit\'e math\'ematiques (mention A.B.)}
{Lyc\'ee Bossuet}{Brive la Gaillarde}{}{}
%-----------------------------------------------------------------------------

%\cvcomputer{Langages}{C++, C, Ruby, Php, CMake}{}{}
%\cvcomputer{Analyse S\'emantique}{Flex Bison}{}{}

%\section{Fameworks, Librairies}
%\cvlistdoubleitem[\Neutral]{Qt}{Boost}

\bigskip
\section{Langues}
\cvlanguage{Anglais}{lu, parl\'e, \'ecrit}{}

\section{Centres d'int\'er\^ets}
%\cvline{Projet Euler}{\small Je participe \`a ce projet qui allie algorithmique et math\'ematiques http://projecteuler.net}
\cvline{Sports}{VTT, Squash, Course \`a pieds}

\renewcommand{\listitemsymbol}{-} % change the symbol for lists
\newpage
\section{D\'etails exp\'eriences}

\cventry{02/13 - , Consultant Ausy pour Orange}{PnS}{}{}{}{
\begin{enumerate}[$\bullet$ ]
\item D\'eveloppement d'un module Apache de duplication de requ\^etes. \em{Mod\_dup} permet de tester un web service candidat \`a une mise en production, en conditions iso-production SANS impact sur la QoS
  \begin{enumerate}
  \item \hfill \url{https://github.com/Orange-OpenSource/mod_dup}
  \end{enumerate}
\item Reprise d'un driver de base de donnn\'ees Cassandra pour le langage PHP
  \begin{enumerate}
  \item Driver \'ecrit en C++, Communication avec PHP via l'interface PDO
  \item \hfill \url{https://github.com/Orange-OpenSource/YACassandraPDO}
  \end{enumerate}
\item Migration des diff\'erentes BDD du web service PHP vers la base Cassandra
\item Animation d'une formation au langage Ruby
\item Int\'egration continue, D\'eveloppement Agile. Fortes contraintes op\'erationnelles. Pns c'est: 140 serveurs, 150k lignes de code, une QoS superieur a 99.99\%, 5000 requ\^etes par seconde, une \'equipe de 20 personnes
\end{enumerate}
\bigskip
}


\cventry{08/12 - 01/13, Consultant Ausy pour SpotImage}{Cartographie}{Conception Impl\'ementation}{}{}{
Développement d'une solution permettant la visualisation et l'édition d'images dont l'objectif est la production de données cartographiées
\begin{enumerate}[$\bullet$ ]
\item Impl\'ementation d'un filtre permettant de corriger des niveaux d'\'el\'evations \'erron\'es
\item D\'eveloppement d'outils de mesures de distances et surfaces sur des mod\`eles d'\'el\'evations.
\item Maintenance applicative
\item Applicatif de plusieurs millions de lignes de code, \'equipe de 8 personnes
\end{enumerate}
\bigskip
}


%\cventry{08/10 - 09/11}{Ensuite Informatique}{D\'eveloppement de processeurs d'analyse de situation en continue}{Conception, impl\'ementation}{Paris}{Environnement: C++, Qt, Flex/Bison, CMake, Git, GNU/Linux}
\cventry{03/12 - 08/12, DoremiLabs}{Cin\'ema Num\'erique, d\'epartement R\&D}{Conception Impl\'ementation}{
}{}{
\begin{enumerate}[$\bullet$ ]
\item D\'eveloppement d'une solution permettant la transmission d'év\`enements en direct dans les salles de cinéma
\item Réception et analyse de flux en DVB-S au format T.S. (Transport Stream) et configuration de la connexion sur un ou plusieurs transpondeurs
\item Re-Engineering de l'ensemble logiciel pour obtenir un support multi-clients. Lecture multi flux dans plusieurs salles avec des configurations différentes (langues / sous-titres)
\item Contraintes temps r\'eel, contexte fortement multi-thread\'e, \'equipe de 2 personnes
\end{enumerate}
\bigskip
}


\cventry{10/11 - 03/12, MargOconseil}{Finance des march\'es, d\'epartement R\&D}{Conception Impl\'ementation}{}{}{
  Développement d'un framework de ré-engineering et génération automatique de code.
  Miles est une solution de re-engineering applicatif qui permet aux clients de s’affranchir de la majorité des problématiques techniques et de se concentrer sur leur métier.
  \begin{enumerate}[$\bullet$ ]
  \item Framework Model-Driven
  \item Programmation orientée aspect.
%  \item Génération d'une librairie dynamique en C++, basée sur le design pattern Mixin-Layers.
  \item Interopérabilité totale entre les languages Ruby AspectJ et C++
  \item Développement d'un convertisseur d'objets Ruby C++ au compile time
  \item Développement en test driven avec une méthodologie agile, \'equipe de 5 personnes
  \end{enumerate}
  \bigskip
}


\cventry{08/10 - 09/11, Ensuite-Info}{Entreprise sp\'ecialis\'ee dans l'analyse situationnelle}{Conception Impl\'ementation}{}{}{
  Réalisation d'un processeur d\'analyse de situation de bout en bout.
  L'analyse de situation est une façon de représenter l'univers qui nous entoure, en éléments compréhensibles et manipulables par une machine.
  \begin{enumerate}[$\bullet$ ]
    \item Implémentation de 2 parseurs de langages dedi\'es \`a l'analyse de situation (Flex/Bison)
    \item  Module d'abstraction de BDD, tests unitaires
    \begin{enumerate}[$\bullet$ ]
    \item Exemple d'utilisation: \em{Alectryon} Outil de veille technologique, le processeur qualifie des URLs en fonction d'un domaine de recherche défini par l'utilisateur et une application tierce présente les résultats tri\'es selon un indice de pertinence attribué par le processeur.
    \item Exemple d'utilisation: \em{Duquenne}: Analyse de diagnostics médicaux, D\'etection de singularités (mise en évidence des traitements les mieux adaptés à une pathologie, détection des comportements anormaux...)
    \end{enumerate}
  \end{enumerate}
}

%% \cventry{KDE}{Contributions logiciels libres}{}{}{}{
%%   \begin{enumerate}[$\bullet$ ]
%%   \item Okular: Lecteur multi-formats de documents de KDE
%%       \begin{enumerate}[$\bullet$ ]
%%       \item Correction de bugs
%%       \item Implémentation de fonctionnalités
%%       \end{enumerate}
%%     \item Kdev-Valgrind: Intégration de valgrind au sein de l'IDE KDevelop
%%       \begin{enumerate}[$\bullet$ ]
%%       \item Restructuration de l'existant
%%       \item Support de l'outil memcheck, affichage des erreurs sous forme d'arbre + interactivités avec le code source
%%       \end{enumerate}
%%     \item Environnement technique: C++, QT, Git, Valgrind, GDB, Emacs
%%   \end{enumerate}
%%   \smallskip
%% }

%% \cventry{Epitech}{Création d'un DVR (Digital Video Recorder)}{Projet Scolaire}{}{}{
%%   \begin{enumerate}[$\bullet$ ]
%%   \item Réaliser un DVR permettant de regarder un flux (Webcam ou fichier vid\'eo), l'enregistrer, contrôler le direct, visionner des enregistrements
%%   \item Réalisation de 2 codecs videos: un utilisant les extensions SSE2 l'autre cod\'e en OpenCL
%%   \item Développement d'une GUI en QT, rendu des images en QOpengl
%%   \item Contexte multitâches où l'optimisation du code fut importante.
%%   \item Environnement technique: C++, QT, OpenCL, OpenCV, SSE2, CMake, Git, Valgrind, GDB, Emacs
%%   \end{enumerate}
%%   \smallskip
%% }

%% \cventry{Epitech}{MyMalloc}{R\'ealisation d'un alloueur dynamique de  memoire}{Projet Scolaire}{}{
%%   \begin{enumerate}[$\bullet$ ]
%%   \item Impl\'ementation de l'algorithme invent\'e par K\&R
%%   \item Support du multit\^aches
%%   \item Environnement technique: C, Makefile
%%   \end{enumerate}
%% \bigskip}

% \section{Extra 1}
%% \cvlistitem{Item 1}
%% \cvlistitem{Item 2}
%% \cvlistitem[+]{Item 3}            % optional other symbol

%% \section{Extra 2}
%% \cvlistdoubleitem[\Neutral]{Item 2}{Item 5}
%% \cvlistdoubleitem[\Neutral]{Item 3}{}

% Publications from a BibTeX file
\nocite{*}
\bibliographystyle{plain}
\bibliography{publications}       % 'publications' is the name of a BibTeX file
\end{document}

%% end of file `template_en.tex'.
